\documentclass[a4paper,12pt]{article}
\usepackage[utf8x]{inputenc}
\usepackage{amssymb}
\usepackage{amsfonts}
\usepackage{mathrsfs}
\usepackage{amsmath}
\usepackage{amsthm}
\usepackage[margin=3cm]{geometry}
\usepackage{times}
\usepackage{graphicx}
\usepackage{enumitem}
\usepackage{fancyhdr}
\usepackage{hyperref}
\usepackage{setspace}
\usepackage{subcaption}
\usepackage{mathtools}

\pagestyle{fancy}
\fancyhf{}
\lhead{Thomas Delaney}
\rhead{Communities in correlation based networks}
\cfoot{\thepage}

\newtheorem{theorem}{Theorem}
\newtheorem{proposition}{Proposition}[section]
\newtheorem{lemma}{Lemma}[section]
\newtheorem{corollary}{Corollary}[section]
\theoremstyle{definition}
\newtheorem{definition}{Definition}[section]

\newcommand{\boldnabla}{\mbox{\boldmath$\nabla$}} % to be used in mathmode
\newcommand{\cbar}{\overline{\mathbb{C}}}% to be used in mathmode
\newcommand{\diff}[2]{\frac{d #1}{d #2}}% to be used in mathmode
\newcommand{\difff}[2]{\frac{d^2 #1}{d #2^2}}% to be used in mathmode
\newcommand{\pdiff}[2]{\frac{\partial #1}{\partial #2}} % to be used in mathmode
\newcommand{\pdifff}[2]{\frac{\partial^2 #1}{\partial #2^2}}% to be used in mathmode
\newcommand{\upperth}{$^{\mbox{\footnotesize{th}}}$}%to be used in text mode
\newcommand{\vect}[1]{\mathbf{#1}}% to be used in mathmode
\newcommand{\curl}[1]{\boldnabla \times \vect{#1}} % to be used in mathmode
\newcommand{\divr}[1]{\boldnabla \cdot \vect{#1}} %to be used in mathmode
\newcommand{\modu}[1]{\left| #1 \right|} %to be used in mathmode
\newcommand{\brak}[1]{\left( #1 \right)} % to be used in mathmode
\newcommand{\comm}[2]{\left[ #1 , #2 \right]} %to be used in mathmode
\newcommand{\dop}{\vect{d}} %to be used in mathmode
\newcommand{\cov}{\text{cov}} %to be used in mathmode
\newcommand{\var}{\text{var}} %to be used in mathmode
\newcommand{\mb}{\mathbf} %to be used in mathmode
\newcommand{\bs}{\boldsymbol} %to be used in mathmode
% Title Page
\title{Communities in correlation based networks}
\author{Thomas Delaney}

\begin{document}

\tableofcontents

\newpage

\section{Motivation}
    \subsection{Finding highly correlated networks}
    We wanted to find highly correlated communities among larger ensembles of neurons. In order to do this, we induced an undirected weighted network in the neural ensemble by measuring the spike count correlations between every pair of neurons. We then used a cutting edge community detection method \cite{humphries} on this network to find any highly correlated communities present in the ensemble.

    \subsection{Within regions or across regions}
    Information in the brain is carried in correlated network activity. Correlations can carry sensory information \cite{cohen1}. Recent findings how that spontaneous behaviours explain correlations in parts of the brain not associated with motor control \cite{stringer}. We wanted to know if the highly correlated communities that we found existed within anatomical regions, or between anatomical regions. That is, did all the members of the community come from one anatomical region, or from many.

    \subsection{Changing timescales}
    The time bin width used when binning spikes into spike counts affects spike count correlation measures \cite{cohen2}. We found that spike count correlations increased as the width of the bin width used increased. However, we also found that the difference between \textit{within-region} and \textit{between-region} correlations decreased as the width of the bin width used increased. 

% we're getting into results here Need more motivation.

\section{Data}
    \subsection{Brain regions}
    We collected from nine different brain regions. They were: ...

\section{Methods}
    \subsection{Binning data}
    The data were divided into time bins of various widths ranging from $0.01$s to $4$s. If the total length of the recording period was not an integer multiple of the time bin width, we cut off the remaining time at the end of the recording period. This period was at most $3.99$s. This is far less than the total recording time of around $1$ hour.

    \subsection{Correlation coefficients}

    \subsubsection{Spike Count Correlation, $r_{SC}$}\label{sec:spike_count_correlation}

    \subsubsection{Separating Correlations \& Anti-correlations}\label{sec:corr_anti_corr}

    \subsection{Network analysis}

    \subsubsection{Correlation networks}

    \subsubsection{Community detection}



\newpage
\bibliography{eight_probe.bbl}

\end{document}
